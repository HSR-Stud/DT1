%Schriftgr�sse, Layout, Papierformat, Gleichungen Linksb�ndig
\documentclass[10pt,twoside,a4paper,fleqn]{article}
\usepackage[left=1cm,right=1cm,top=1cm,bottom=1cm,includeheadfoot]{geometry}
\usepackage[latin1]{inputenc}
\usepackage[ngerman]{babel,varioref}
%Packages - Von LinAlg Formelsammlung kopiert! �berpr�fen ob die alle notwendig sind
\usepackage{amsmath,amssymb,fancybox,graphicx,color,lastpage,wrapfig,fancyhdr,hyperref,verbatim,paralist}
\usepackage{listings}
%Package f�r Bilder im Fliesstext
\usepackage{multicol}

%Package f�r Tabelle im Querformat
\usepackage{rotating}

\raggedright
%Titel, Autor
\newcommand{\titleinfo}{Digitaltechnik DT1 - Zusammenfassung}
\newcommand{\authorinfo}{G. Koeppel, D. Koeppel, L. Leuenberger}
\newcommand{\versioninfo}{v2.0 / Januar 2013}

%pdf info
\hypersetup{pdfauthor={\authorinfo},pdftitle={\titleinfo},colorlinks=false}
%linkbordercolor=white
\author{\authorinfo}
\title{\titleinfo}

%Kopf- und Fusszeile
\pagestyle{fancy}
\fancyhf{}
%Linien oben und unten
\renewcommand{\headrulewidth}{0.5pt} 
\renewcommand{\footrulewidth}{0.5pt}

\fancyhead[L]{\titleinfo{ }\tiny{(\versioninfo)}}
%Kopfzeile rechts bzw. aussen
\fancyhead[R]{Seite \thepage { }von \pageref{LastPage}}
%Fusszeile links bzw. innen
\fancyfoot[L]{\footnotesize{\authorinfo}}
%Fusszeile rechts bzw. ausen
\fancyfoot[R]{\footnotesize{\today}}

%Document Anfang
\begin{document}

\section{Von Analog zu Digital}
	\begin{minipage}[c]{5 cm}
		\includegraphics[width=0.9\textwidth]{pics/von_analog_zu_digital.jpg}
	\end{minipage}
	\begin{minipage}[c]{13 cm}
		\begin{minipage}[c]{4 cm}
			Abtasttheorem:
			\newline\newline\newline
			Amplitudenaufl�sung:
			\newline\newline\newline\newline\newline\newline
			Quantisierungsfehler:
			\newline\newline
			dynamischer Bereich AD-Wandler:
		\end{minipage}
		\begin{minipage}[c]{9 cm}
			Nyquist-Shannon besagt, dass ein Signal mit $f_{max}$ mit mindestens einer Frequenz von $2*f_{max}$ abgetastet werden muss, um das Ursprungssignal wieder herzustellen.\\
			Die Anzahl abz�hlbarer Amplitudenwerte (Quantisierungsstufen) bestimmt die Aufl�sung eines AD-Wandlers. Diese wird in der Regel in Bits angegeben. Je kleiner der Bereich aufgeteilt ist, desto kleiner ist der Abstand $\Delta A$ zwischen zwei benachbarten Amplitudenwerten.\\
			Anz. Quantisierungsstufen: $n_{q} = \frac{V_{max} - V{min}}{\Delta A} =$ n \\
			Bei linearer Quantisierung ist der Quantisierungsfehler (Quantisierungsrauschen) max. $\frac{\Delta A}{2}$\\
			\newline
			$DR_{ADC} = U_{max} / U_{min}$\\
			Faustformel: $DR_{ADC}[dB] = 6 * \# Bit$\\
		\end{minipage}
	\end{minipage}

\section{Bin"are Multiplikation}
	\begin{multicols}{2}
		\includegraphics[width=0.7\columnwidth]{pics/multiplikation}
		\columnbreak
		
		Multiplikation von zwei n-Bit W�rtern\\
		Gr�sstes zu erwartendes Ergebnis:\\
		\ \newline
		$E=(2^n-1)*(2^n-1)=2^{2n}-2^{n+1}+1\leq2^{2n}-1$\\
		\ \newline\newline
		Ergebnis kann also max. 2n Bits lang sein!
	\end{multicols}

\section{CMOS Beschaltung}
	\subsection{Beispiel: Inverter}
		\begin{multicols}{2}
			\includegraphics[width=0.5\textwidth]{pics/cmos_transistorenL.png}
			\columnbreak
			
			\includegraphics[width=0.5\textwidth]{pics/cmos_transistorenR.png}\newline
			PMOS leitet bei Low-Pegel\\
			NMOS leitet bei High-Pegel\\
			\columnbreak
		\end{multicols}
	
\section{RAM}
	\begin{multicols}{2}
		\subsection{SRAM}
		\includegraphics[width=0.5\columnwidth]{pics/sram.png}\newline\newline
		Vorteil: Zustand bleibt, solange Strom vorhanden
			
		\subsection{DRAM}
		\includegraphics[width=0.3\columnwidth]{pics/dram.png}\newline
		Vorteil: Grosse Dichte einzelner Zellen\\
		Nachteile: \\-Leckstrom Kapazit"at\\
				   -Refresh\\
				   -Steuerlogik\\

		
	\end{multicols}
			

		
	

\section{Zahlendarstellungen und Codes}

\subsection{Zahlensysteme}
	\subsubsection{Zahlensysteme ohne festen Stellenwert}
		\begin{compactitem}
			\item R�misches Zahlensystem (keine Null, kein fester Stellenwert, schlechte Unterscheidung)
		\end{compactitem}
		
	\subsubsection{Zahlensysteme mit festem Stellenwert}
		\begin{multicols}{3}
			\begin{compactitem}
				\item Babylon (Basis B = 60)
				\item Maya (Basis B = 20)
			\end{compactitem}
			\columnbreak
			\begin{compactitem}
				\item Dezimal (Basis B = 10)
				\item Bin�r (Basis B = 2)
			\end{compactitem}
			\columnbreak
			\begin{compactitem}
				\item Octal (Basis B = 8)
				\item Hexadezimal (Basis B = 16)
			\end{compactitem}
		\end{multicols}
		
		\begin{minipage}{19 cm}
			Die Wertigkeit des Symbols h�ngt von seiner Position innerhalb der Symbolkette ab:\\
			\newline
			\begin{minipage}[c]{3 cm}
				$z=\sum\limits_{k=0}^{n-1}a_k*B^k$
			\end{minipage}
			\begin{minipage}[c]{6 cm}
				z: Wert der Zahl (im Dezimalsystem)\\
				a: Nennwert der Ziffer\\
				B: Basis des Zahlensystems\\
				n: Stellenanzahl\\
			\end{minipage}
			\begin{minipage}[c]{2.4 cm}
				Bsp: $4156.78=$\\
			\end{minipage}
			\begin{minipage}[c]{6.6 cm}
				$4*10^3+1*10^2+5*10^1+6*10^0$\\
				$+7*10^{-1}+8*10^{-2}$\\
			\end{minipage}
		
			Auch gebrochene Zahlen k�nnen nach dem gleichen Muster bin�r dargestellt werden. Wichtig ist, dass das Komma immer an einer festen Stelle steht (Festkommadarstellung). Definiert ist, dass eine Bin�rzahl 8 Ziffern (n) vor dem Komma und 4 Ziffern (m) nach dem Komma besitzt. Eine gebrochene Bin�rzahl sieht dann so aus:\\
			\newline
			$z_2=a_{n-1}a_{n-2}\dots a_1a_0.a_{-1}a_{-2}\dots a_{-m+1}a_{-m}$\\
			\newline
			Der gesuchte Zahlenwert im Dezimalsystem wird dann folgendermassen berechnet:\\
			\newline
			$z_{10}=a_{n-1}*B^{n-1}+a_{n-2}*B^{n-2}+\dots +a_0*B^0+a_{-1}*B^{-1}+a_{-2}*B^{-2}+\dots +a_{-m+1}*B^{-m+1}+a_{-m}*B^{-m}$
		\end{minipage}

\subsection{Gebr�uchliche polyadische Zahlensysteme}
	\begin{tabular}{|l|l|l|l|l|}
		\hline
			System & Basis & Stellenwerte & Ziffern & Beispiel\\
		\hline
		\hline
			Bin�r & 2 & $\dots$ $2^2$ $2^1$ $2^0$ $\dots$ & 0, 1 & $110_{(2)}=6_{(10)}$\\
		\hline
			Oktal & 8 & $\dots$ $8^2$ $8^1$ $8^0$ $\dots$ & 0, 1, 2, 3, 4, 5, 6, 7 & $273_{(8)}=187_{(10)}$\\
		\hline
			Dezimal & 10 & $\dots$ $10^2$ $10^1$ $10^0$ $\dots$ & 0, 1, 2, 3, 4, 5, 6, 7, 8, 9 & $192_{(10)}=192_{(10)}$\\
		\hline
			Hexadezimal & 16 & $\dots$ $16^2$ $16^1$ $16^0$ $\dots$ & 0, 1, 2, 3, 4, 5, 6, 7, 8, 9, A, B, C, D, E, F & $2$AFF$_{(16)}=11007_{(10)}$\\
		\hline
	\end{tabular}

\subsection{Begriffe im Zusammenhang mit dem bin�ren Zahlensystem}
	\begin{compactitem}	
		\item 
			\begin{tabbing}
				xxxxxxxxxxx\=xxxxxxxxxxxxxxxxxxxxxxxxxxxxxxxxxxxxx\kill	
				Bit (b): \>
							Binary Digit: Kleinsm�gliche Speichereinheit in der Digitaltechnik. Kann zwei m�gliche Zust�nde\\
				 		\>	annehmen: 0 und 1
			\end{tabbing}
		\item 
			\begin{tabbing}
				xxxxxxxxxxx\=xxxxxxxxxxxxxxxxxxxxxxxxxxxxxxxxxxxxx\kill	
				Byte (B): \>
							Einheit von 8 Bits. Auch genannt Oktett: 1 Oktett = 1 Byte = 8 Bit. Byte ist die Standartbezeichnung\\
						\>	von Speicherkapazit�ten und Datenmengen.
			\end{tabbing}
		\item 
			\begin{tabbing}
				xxxxxxxxxxx\=xxxxxxxxxxxxxxxxxxxxxxxxxxxxxxxxxxxxx\kill	
				Nibble: \>
							Bin�rzahlen in Gruppen von 4 Bits
			\end{tabbing}
		\item 
			\begin{tabbing}
				xxxxxxxxxxx\=xxxxxxxxxxxxxxxxxxxxxxxxxxxxxxxxxxxxx\kill	
				MSB: \>
							Most Significant Bit. Bit mit h�chster Wertigkeit, steht ganz links im bin�ren Wort
			\end{tabbing}
		\item 
			\begin{tabbing}
				xxxxxxxxxxx\=xxxxxxxxxxxxxxxxxxxxxxxxxxxxxxxxxxxxx\kill	
				LSB: \>
							Least Significant Bit. Bit mit tiefster Wertigkeit, steht ganz rechts im bin�ren Wort
			\end{tabbing}
	\end{compactitem}
	
\subsection{Umwandlung von Dezimalzahlen}
	Beispiel f�r die Umwandlung der Zahl $109.78125_{(10)}$:
	\begin{tabular}{|lllllll|lllllll|lllllll|}
		\hline
			\multicolumn{7}{|l|}{Dezimal zu Bin�r} & \multicolumn{7}{|l|}{Dezimal zu Oktal} & \multicolumn{7}{|l|}{Dezimal zu Hexadeximal}\\
		\hline
		\hline
			109 & : & 2 & = & 54 & Rest: & 1 &
			109 & : & 8 & = & 13 & Rest: & 5 & 
			109 & : & 16 & = & 6 & Rest: & D \\
			
			54 & : & 2 & = & 27 & Rest: & 0 &
			13 & : & 8 & = & 1 & Rest: & 5 & 
			6 & : & 16 & = & 0 & Rest: & 6 \\
			
			27 & : & 2 & = & 13 & Rest: & 1 &
			1 & : & 8 & = & 0 & Rest: & 1 & 
			\multicolumn{7}{|l|}{} \\
						
			13 & : & 2 & = & 6 & Rest: & 1 &
			\multicolumn{7}{|l|}{} & 
			\multicolumn{7}{|l|}{} \\
			
			6 & : & 2 & = & 3 & Rest: & 0 &
			\multicolumn{7}{|l|}{} & 
			\multicolumn{7}{|l|}{} \\
			
			3 & : & 2 & = & 1 & Rest: & 1 &
			\multicolumn{7}{|l|}{} & 
			\multicolumn{7}{|l|}{} \\
			
			1 & : & 2 & = & 0 & Rest: & 1 &
			\multicolumn{7}{|l|}{} & 
			\multicolumn{7}{|l|}{} \\
		\hline
			0.78125 & * & 2 & = & 0.5625 & + & 1 &
			0.78125 & * & 8 & = & 0.25 & + & 6 &
			0.78125 & * & 16 & = & 0.5 & + & C \\
			
			0.5625 & * & 2 & = & 0.125 & + & 1 &
			0.25 & * & 8 & = & 0 & + & 2 &
			0.5 & * & 16 & = & 0 & + & 8 \\
			
			0.125 & * & 2 & = & 0.25 & + & 0 &
			\multicolumn{7}{|l|}{} &
			\multicolumn{7}{|l|}{} \\
			
			0.25 & * & 2 & = & 0.5 & + & 0 &
			\multicolumn{7}{|l|}{} &
			\multicolumn{7}{|l|}{} \\
			
			0.5 & * & 2 & = & 0 & + & 1 &
			\multicolumn{7}{|l|}{} &
			\multicolumn{7}{|l|}{} \\
		\hline
		\hline
			\multicolumn{7}{|l|}{$109_{(10)}=1101101.11001_{(2)}$} & \multicolumn{7}{|l|}{$109_{(10)}=155.62_{(8)}$} & \multicolumn{7}{|l|}{$109_{(10)}=6$D$.$C$8_{(16)}$}\\
		\hline
	\end{tabular}
	
	
	
%\section{Schaltalgebra}
\subsection{Symbolik}
	Folgende Symbole werden verwendet:\\
	\begin{multicols}{4}
		$\neg=^-=$NOT
	\columnbreak
	
		$\vee=+=$OR
	\columnbreak
	
		$\wedge=*=$AND
	\columnbreak
	
		$\oplus=\$=$(E)XOR
	\end{multicols}
	
\subsection{Rechenregeln}
	\begin{tabular}{llll}
		Verkn"upfung mit 0 & $ a \vee 0 = a $ & $ a \wedge 0 = 0 $ & $ a \oplus 0 = a $\\
		Verkn"upfung mit 1 & $ a \vee 1 = 1 $ & $ a \wedge 1 = a $ & $ a \oplus 1 = \overline{a} $ \\
		Verkn. mit sich selbst & $ a \vee a = a $ & $ a \wedge a = a $ & $ a \oplus a = 0 $ \\
		Verkn. mit Inversem & $ a \vee \overline{a} = 1 $ & $ a \wedge \overline{a} = 0 $ & $ a \oplus \overline{a} = 1 $ \\
		\\
		Kommutativgesetz & $ a \vee b = b \vee a $ & $ a \wedge b = b \wedge a $ & $ a \oplus b = b \oplus a $\\
		Assioziativgesetz & $ (a \vee b) \vee c = a \vee (b \vee c) $ & $ (a \wedge b) \wedge c = a \wedge (b \wedge c) $ & $ (a \oplus b) \oplus c = a \oplus (b \oplus c) $ \\
		Distributivgesetz & $ a \wedge (b \vee c) = (a \wedge b) \vee (a \wedge c) $ & $ a \vee (b \wedge c) = (a \vee b) \wedge (a \vee c) $ & $ a \wedge (b \oplus c) = (a \wedge b) \oplus (a \wedge c) $ \\	
		\end{tabular}
		
\subsubsection{Vereinfachungen \& Minimierungen}
	\begin{multicols}{4}
		$ a \vee (a \wedge b) = a $ \\
		$ a \wedge (a \vee b) = a $ \\
		$ (a \vee b) \wedge (a \vee \neg b) = a $ 
	\columnbreak
	
		$ (a \wedge \overline{b}) \vee b = a \vee b $ \\
		$ (a \vee \overline{b}) \wedge b = a \wedge b $ \\
		$ (a \wedge b) \vee (a \wedge \neg b) = a $ 
	\columnbreak
	
		$ (a \wedge \overline{b}) \oplus b = a \vee b $ \\
		$ (a \oplus \overline{b}) \wedge b = a \wedge b $ \\
		$ (a * b) + (a * \neg b) = a $ 
	\columnbreak
		
		$ (a \wedge b) \vee (a \wedge \overline{b}) = a $\\	
		$ (a \vee b) \wedge (a \vee \overline{b}) = a $ \\
		$ (a + b) * (a + \neg b) = a $ 
	\end{multicols}

%\subsection{Minimierungsregeln}
%\begin{multicols}{4}
%	$ (a \vee b) \wedge (a \vee \neg b) = a $ \\
%	\columnbreak
%	$ (a \wedge b) \vee (a \wedge \neg b) = a $ \\
%	\columnbreak
%	$ (a * b) + (a * \neg b) = a $ \\
%\columnbreak
%	$ (a + b) * (a + \neg b) = a $ \\
%\end{multicols}


\subsection{Shannon und DeMorgan}
	\begin{tabular}{lll}
		Ursprungsschaltung: & Shannon & DeMorgan\\
		\includegraphics[width=0.3\textwidth]{pics/shanonursprung} & 
		\includegraphics[width=0.3\textwidth]{pics/demorganende} &
		\includegraphics[width=0.3\textwidth]{pics/shanonende}\\
		Shannon & DeMorgan & \\
		$\neg f(a, b, c, ...z; \wedge, \vee) = f(\neg a, \neg v, \neg c, ... \neg z; \wedge, \vee)$ & $\neg(a \vee b) = \neg a \wedge \neg b$ & 
		  $\neg(a \wedge b) = \neg a \vee \neg b $ \\ 
	\end{tabular}
% Die Negation einer beliebigen log. Funktion f erh"alt man, wenn man alle Variablen durch ihre Negation ersetzt und die Operatoren vertauscht. \\

\subsection{Normalformen: KDNF und KKNF}
	\begin{minipage}{10cm}
		Ausgangslage:\\
		\begin{tabular}{|l|l|l|l|l||l||l|}
			\hline	
				Dezimal & $x_2$ & $x_1$ & $x_0$ & y & KDNF & KKNF\\	
			\hline
			\hline
				0 & 0 & 0 & 0 & 0 & & $x_2 \vee x_1 \vee x_0$ \\
			\hline	
				1 & 0 & 0 & 1 & 1 &	$\overline{x_2} \wedge \overline{x_1} \wedge x_0$ & \\
			\hline
				2 & 0 & 1 & 0 & 0 & & $x_2 \vee \overline{x_1} \vee x_0$  \\
			\hline
				3 & 0 & 1 & 1 & 1 &	$\overline{x_2} \wedge x_1 \wedge x_0$ & \\
			\hline
				4 & 1 & 0 & 0 & 0 & & $\overline{x_2} \vee x_1 \vee x_0$ \\
			\hline
				5 & 1 & 0 & 1 & d & $[x_2 \wedge \overline{x_1} \wedge x_0]$ & $[\overline{x_2} \vee x_1 \vee \overline{x_0}]$  \\
			\hline
				6 & 1 & 1 & 0 & 1 & $x_2 \wedge x_1 \wedge \overline{x_0}$ &  \\
			\hline
				7 & 1 & 1 & 1 & 1 & $x_2 \wedge x_1 \wedge x_0$ &\\
			\hline
		\end{tabular}
\end{minipage}
\begin{minipage}{8cm}
Kanonisch disjunktive Normalform \lbrack KDNF\rbrack: Es werden nur diejenigen Zeilen der Wahrheitstabelle aufgef"uhrt, deren Funktionswert 1 oder d ist. \\
$y=(\overline{x_2} \wedge \overline{x_1} \wedge x_0) \vee (\overline{x_2} \wedge x_1 \wedge x_0) \vee [x_2 \wedge \overline{x_1} \wedge x_0] \vee (x_2 \wedge x_1 \wedge \overline{x_0}) \vee (x_2 \wedge x_1 \wedge x_0)$ \\
\\
Kanonisch konjunktive Normalform \lbrack KKNF\rbrack: 
Es werden nur diejenigen Zeilen der Wahrheitstabelle aufgef"uhrt, deren Funktionswert 0 oder d ist. \\
$y=(x_2 \vee x_1 \vee x_0) \wedge (x_2 \vee \overline{x_1} \vee x_0) \wedge (\overline{x_2} \vee x_1 \vee x_0) \wedge [\overline{x_2} \vee x_1 \vee \overline{x_0}]$ \\
\end{minipage}
\subsection{Karnaugh-Diagramm}
	\subsubsection{Arbeiten mit dem KV-Diagramm}
\begin{compactitem}
	\item 1) \ \ Aufstellen der Wahrheitstabelle.\\
	\item 2) \ \ "Ubertragen der Werte der Wahrheitstabelle ins KV Diagramm.\\
	\item 3) \ \ M"oglichst grosse Gruppen a $2^n$ Felder bilden (k"onnen auch "uber die Ecken gehen).\\
	\item 4)
	\begin{tabular}{ll}
		Kanonisch disjunktive Normalform: & Kanonisch konjunktive Normalform: \\
		Gruppen von Feldern mit Wert 1 oder d & Gruppen von Feldern mit Wert 0 oder d\\
		Primimplikanten: AND Verkn"upfung & Primimplikanten: OR Verkn"upfung\\
		OR Verkn"upfung aller Primimplikanten & AND Verkn"upfung aller Primimplikanten\\
	\end{tabular}
\end{compactitem}
\newpage
\subsection{Karnaugh-Diagramm}
	\begin{multicols}{3}
		\includegraphics[width=0.3\textwidth]{pics/kv/2erKV} 
	\columnbreak
	
		\includegraphics[width=0.3\textwidth]{pics/kv/3erKV}
	\columnbreak
	
		\includegraphics[width=0.3\textwidth]{pics/kv/4erKV}
	\end{multicols}
	
%	\subsubsection{Arbeiten mit dem KV-Diagramm}
%		\begin{compactitem}
%			\item 1) \ \ Aufstellen der Wahrheitstabelle.\\
%			\item 2) \ \ "Ubertragen der Werte der Wahrheitstabelle in KV Diagramm.\\
%			\item 3) \ \ M"oglichst grosse Gruppen a $2^n$ Felder bilden.\\
%			\item 4)
%				\begin{tabular}{ll}
%					Kanonisch disjunktive Normalform: & Kanonisch konjunktive Normalform: \\
%					Gruppen von Feldern mit Wert 1 oder d & Gruppen von Feldern mit Wert 0 oder d\\
%					Primimplikanten: AND Verkn"upfung & Primimplikanten: OR Verkn"upfung\\
%					OR Verkn"upfung aller Primimplikanten & AND Verkn"upfung aller Primimplikanten\\
%				\end{tabular}
%		\end{compactitem}
%\section{VHDL}
	Die vollst�ndige Beschreibung des Designs besteht aus:\\
	\begin{itemize}
	\setlength{\itemsep}{1pt}
  \setlength{\parskip}{0pt}
  \setlength{\parsep}{0pt}
		\item Bibliothekenbeschreibung
		\item Schnittstellenbeschreibung
		\item Architekturbeschreibung
	\end{itemize}
	\subsection{Key Concepts}
		\begin{tabular}{ll}
			Key Concept I: & Schaltungshierarchie und Verbindung von Sub-Bl�cken (hierarchy and connectivity).\\
			Key Concept II: & Nebenl�ufige (concurrent) Prozesse und Prozess-Interaktion.\\
			Key Concept III: & Modellierung des elektrischen Verhaltens von Signalen.\\
			Key Concept IV: & Event-Based time: Simulationsmodell, das auf Events und nicht auf kontinuierlicher Zeit beruht.\\
			Key Concept V: & Parametrisierung von Modellen.
		\end{tabular}
	\subsection{Bibliotheken}
		\begin{tabular}{ll}
			work & Default-Bibliothek des Benutzers\\
			std & Enth�lt standard (Vordefinierte Datentypen und Funktionen und\\
			& textio (Dateioperationen)\\
			ieee & std\_logic\_1164: Datentypen f�r mehrwertiges Logiksystem
		\end{tabular}
		\lstinputlisting[language=VHDL,tabsize=2]{code/header.vhd}
	\subsection{Schnittstellenbeschreibung (Entity)}
		Die einzelnen Bl�cke einer VHDL-Beschreibung kommunizieren �ber ihre Schnittstellen miteinander. Die Kommunikationskan�le nach aussen sind die sogenannten Ports. F�r diese werden in der Schnittstellenbeschreibung Name, Signalflussrichtung und Datentyp festgelegt. Mit der Signalflussrichtung werden Eing�nge (IN), Ausg�nge (OUT) und bidirektionale Ports (INOUT) unterschieden.
		\lstinputlisting[language=vhdl,tabsize=2]{code/entity.vhd}
	\subsection{Architekturbeschreibung}
		Die Architektur legt die Funktion eines Blocks fest. In VHDL wird ein Block als Entity oder auch als Component bezeichnet. Die Architecture besteht aus folgenden Elementen:
		\subsubsection{Signaldeklaration}
		 Hier werden die Signale die Innerhalb der Architektur verwendet werden deklariert.
		 \lstinputlisting[language=vhdl,tabsize=2]{code/signal.vhd}
		\subsubsection{Komponentendeklaration}
			Mit Hilfe des Schl�sselwortes COMPONENT erfolgt die Deklaration von Komponenten f�r die m�glicherweise mehrfache Instanziierung (Platzierung von Komponenten) in dar�ber liegenden hierarchischen Ebenen.\\
			Geschrieben wird das ganze beinahe gleich wie eine ENTITY, nur wird das wort ENTITY durch COMPONENT ersetzt.
		\subsubsection{Instanzierung}
			Hier werden die Bl�cke (components) platziert und "`verdrahtet"'.
			\begin{multicols}{2}
			\lstinputlisting[language=vhdl,tabsize=2]{code/instanzierung.vhd}
			\end{multicols}
	\subsection{Signal Typen}
		\begin{tabular}{ll}
			in: & Eingangssignal. Darf nur rechts stehen.\\
			out: & Ausgangssignal. Darf nur links stehen.\\
			buffer: & Ausgangssignal. Darf auch rechts stehen, aber problematisch.\\
			inout: & Bidirektionales Signal, in Verbindung mit Typ std\_logic.\\
		\end{tabular}
		\\
		Alle Signalzuweisungen und alle Prozesse laufen parallel zueinander. Signalzuweisungen sind immer aktiv. Signale k�nnen auf verschiedene Arten zugewiesen werden:
		\begin{multicols}{2}
		\lstinputlisting[language=vhdl,tabsize=2]{code/signalzuweisung.vhd}
		\end{multicols}
		
\begin{center}
\includegraphics[width=0.5\textwidth]{pics/designprocess}
\end{center}

\subsection{Prozesse}
	\begin{itemize}
		\setlength{\itemsep}{1pt}
  	\setlength{\parskip}{0pt}
  	\setlength{\parsep}{0pt}
		\item Prozesse werden durch �nderungen an den Signalen in der Sensitivit�tsliste (im Prozess-Kopf) aktiviert und ausgef�hrt.
		\item Prozesse werden parallel abgearbeitet
		\item Innerhalb des Prozesses werden Anweisungen sequentiell abgearbeitet
		\item Selektive und konditionale Signalzuweisungen sind verboten.
		\item Unbedingte Signalzuweisung erlaubt. Aktualisierung aller Signale geschieht immer erst am Prozessende!
		\item G�ltig ist letzte Zuweisung
	\end{itemize}
	
	\lstinputlisting[language=vhdl,tabsize=2]{code/process.vhd}

\subsection{VHDL Code Example}
	\begin{center}
		\includegraphics[width=0.5\textwidth]{pics/vhdlcodeexampleschematic.png}
	\end{center}
	\begin{multicols}{2}
		\lstinputlisting[language=vhdl,tabsize=2]{code/vhdlcodeexample.vhd}
	\end{multicols}		
%\section{Realisierungs Methoden}
\subsection{ROM}
Mit einem ROM lassen sich sich kombinatorische Schaltungen in Form einer Look up Table realisieren.\\
\begin{itemize}
	\setlength{\itemsep}{1pt}
  \setlength{\parskip}{0pt}
  \setlength{\parsep}{0pt}
  
	\item Eingangsvariablen = Adresse\\
	\item Speicherwert = Ausgang (programmierbar)\\
\end{itemize}

\begin{itemize}
	\setlength{\itemsep}{1pt}
  \setlength{\parskip}{0pt}
  \setlength{\parsep}{0pt}
  
	\item ROM $\rightarrow$ Bei der Herstellung programmierter Speicher\\
	\item PROM $\rightarrow$ Programmierbares ROM (Fuses)\\
	\item EPROM $\rightarrow$ L"oschbares PROM (UV Licht)\\
	\item EEPROM $\rightarrow$ Elektrisch l"oschbares PROM\\
	\item Flash $\rightarrow$ Blockweise beschreibbares EEPROM (schneller)\\
\end{itemize}

\subsection{PLD}
Programmierbares Device aus AND und OR-Matrix, mindestens eine Matrix programmierbar.
\begin{itemize}
	\setlength{\itemsep}{1pt}
  \setlength{\parskip}{0pt}
  \setlength{\parsep}{0pt}
  
	\item PAL $\rightarrow$ OR-Matrix fest, AND-Matrix programmierbar, Fuses\\
	\item PLA $\rightarrow$ OR und AND Matrix frei programmierbar, Fuses\\
	\item GAL $\rightarrow$ Wie PLA plus programmierbare Ausgangsnetzwerke (Tristate), EEPROM\\
\end{itemize}
F�r Funktionen die als DNF vorliegen geeignet, heute gr�sstenteils von CPLD und FPGA verdr�ngt.\\

\subsection{CPLD}
\begin{itemize}
	\setlength{\itemsep}{1pt}
  \setlength{\parskip}{0pt}
  \setlength{\parsep}{0pt}
  
	\item Verbund PLD Makrozellen die mit Bussen verbunden sind, Speicherung der Konfiguration in Flash.\\
	\item	Durch regelm"assige Struktur sind Signallaufzeiten vorhersagbar.\\
	\item Wegen grosser Zahl an Logikbl�cken sehr gut f�r parallele Prozesse geeignet.\\
\end{itemize}

\subsection{FPGA}
2D-Array von Logikbl"ocken, die �ber Routing Kanal und Schaltmatrizen miteinander und mit I/O verbunden werden.\\
\begin{itemize}
	\setlength{\itemsep}{1pt}
  \setlength{\parskip}{0pt}
  \setlength{\parsep}{0pt}
  
	\item Logikblock (LogicCell) $\rightarrow$ Lookuptable mit D-FlipFlop, kann beliebige Funktionen ausf"uhren
	\item Schaltmatrizen $\rightarrow$ programmierbare Verbindungen
	\item Makrozellen $\rightarrow$ Feste Funktionen z.B. Memory, Clock Managment...
\end{itemize}
Die Konfiguration wird im RAM gespeichert (fl"uchtig). D.h. bei jedem Boot muss der Code von einem Festspeicher geladen werden.\\

\subsection{Xilinx Spartan 3}
\begin{itemize}
	\setlength{\itemsep}{1pt}
  \setlength{\parskip}{0pt}
  \setlength{\parsep}{0pt}

	\item Logic Cell (LC): Kleinste Einheit, enth"alt LUT mit 4 Eing"angen und eime D-FlipFlop. LUT kann als 16x1 bit SRAM oder Schieberegister 		konfiguriert werden. Zus"atzlich pro LC CarryLogic und MUX.
	\item Slice: 1Slice = 2 Logic Cell
	\item Configurable logic bloc (CLB): 1 CLB = 4 Slices = 8 Locic Cells. \\
	Inerhalb dieser Einheit existieren spezifische Verbindungsstrukturen.
\end{itemize}

\includegraphics[width=0.8\textwidth]{pics/fpgastruct}


\subsection{Semicustom IC}
\begin{itemize}
	\setlength{\itemsep}{1pt}
  \setlength{\parskip}{0pt}
  \setlength{\parsep}{0pt}
  
	\item Mikrozellen aus p- und n-FETs werden durch Verdrahtung zu Gates.\\
	\item Gates k"onnen durch Verdrahtungskan"ale verbunden werden.\\
	\item Standardfunktionen k"onnen mit IP-Cores implementiert werden.\\
\end{itemize}	
	
\subsection{Fullcustom IC}
V�llig kundenspezifische ICs, oft werden IP-Cores f�r Standardfunktionen verwendet. Digitale und analoge Komponenten auf einem IC m�glich. Voll auf Anwendung anpassbare Eigenschaften (Stromverbrauch, Gr"osse, Geschwindigkeit etc.).\\

\subsection{Vergeichstabelle}

\begin{center}
	\includegraphics[width=0.60\textwidth]{pics/devicecomparetables}
	
	\begin{tabular}{|l|c|c|c|c|c|c|}
		\hline
		Kriterien & Standard Bauteile & ROM & PLD & FPGA & Semicustom & Fullcustom \\
		\hline
		Machbarkeit & ++ & - - & - - & + & + & +++ \\
		\hline
		Zeit Realisiertung & + & ++ & ++ & ++ & - & - - \\
		\hline
		Iterationszeit & - & ++ & ++ & ++ & - & - - \\
		\hline
		NRE & ++ & + & + & + & - & - - -\\
		\hline
		St"uckpreis & - - & + & + & - & + & +++ \\
		\hline
	\end{tabular}
\end{center}
	
%\section{Logische Gatter}

\subsubsection{Ralisierungsaufwand}
Anzahl Transistoren ist ein m"ogliches Mass f"ur Komplexit"at (Marketing). In der Technik spricht man von Gatter"aquivalent: 1 Gatter"aquivalent entspricht 4 Transistoren.

\subsection{Anzahl Schaltfunktionen}
Permutation: $F= {2^2}^n$ F: Anzahl m"oglicher Schaltfunktionen, n: Anzahl Eing"ange

	\subsection{Verhalten logischer Gatter}
		\begin{multicols}{2}
			\subsubsection{St"orabstand}
				High-Pegel: $ U_{nH} = U_{aHmin} - U_{eHmin} $\\
				Low-Pegel: $ U_{nL} = U_{eLmax} - U_{aLmax} $\\
				
		%\end{multicols}

		%\begin{multicols}{2}
			\subsubsection{Propagation delay Time Anstiegs- bwz. Abstiegszeit}
				Zeit zwischen 50\% von $V_{max}$ am Eingang und 50\% von $V_{max}$ am Ausgang.\\
				$t_{pd}=\frac{t_{pLH}+t{pHL}}{2}$\\
				
			\subsubsection{Transition delay Time (Verz"ogerungszeit)}
				Zeit zwischen 10\% und 90\% von $V_{max}$.\\
				$t_{tLH}$: Transition time low to high.\\
				$t_{tHL}$: Transition time high to low.\\
				\includegraphics[width=0.2\textwidth]{pics/delay}
				\includegraphics[width=0.2\textwidth]{pics/Pegelbereiche_Stoerabstand}
		\end{multicols}
		
		
\subsection{Aufbau logischer Gatter}

%\newpage
\begin{sidewaystable}
\begin{tabular}{|c|c|c|c|c|c|c|c|c|}
\hline
Funktion & Buffer & NOT & AND & NAND & OR & NOR & EXOR & XNOR\\
& & Nicht & Und & Nicht Und & Oder & Nicht Oder & Exklusiv Oder & Nicht Ex. Oder\\
& & Inverter & Konjunktion & & Disjunktion & & Antivalenz & "Aquivalenz \\
\hline
Formel & a & $ \overline a $ & $ a \cdot b $ & $ \overline{a \cdot b} $ & $ a + b $ & $ \overline{a + b} $ & $ a \oplus b $ & $ \overline{a \oplus b} $\\
& a & $ \overline a $ & $ a \wedge b $ & $ \overline{a \wedge b} $ & $ a \vee b $ & $ \overline{a \vee b} $ & $ a \not= b $ & $ \overline{a \not= b} $ \\
& a & !a & $ a \& b $ & $ !(a \& b) $ & a\#b & !(a\#b) & a\$b & !(a\$b) \\
& & & & & & & $ a \veebar b $ & $ \overline{a \veebar b} $\\
\hline
& & & & & & & &\\
& \includegraphics[width=0.08\textwidth]{pics/gates_symbol/buffer} & \includegraphics[width=0.08\textwidth]{pics/gates_symbol/not} & \includegraphics[width=0.08\textwidth]{pics/gates_symbol/and} & \includegraphics[width=0.08\textwidth]{pics/gates_symbol/nand} & \includegraphics[width=0.08\textwidth]{pics/gates_symbol/or} & \includegraphics[width=0.08\textwidth]{pics/gates_symbol/nor} & \includegraphics[width=0.08\textwidth]{pics/gates_symbol/exor} & \includegraphics[width=0.08\textwidth]{pics/gates_symbol/xnor} \\
\hline
(0,0) & 0 & 1 & 0 & 1 & 0 & 1 & 0 & 1\\
(0,1) &   &   & 0 & 1 & 1 & 0 & 1 & 0\\
(1,0) & 1 & 0 & 0 & 1 & 1 & 0 & 1 & 0\\
(1,1) &   &   & 1 & 0 & 1 & 0 & 0 & 1\\
\hline
KDNF & \#(1) & \#(0) & \#(3) & \#(0,1,2) & \#(1,2,3) & \#(0) & \#(1,2) & \#(0,3) \\
KKNF & \&(0) & \&(1) & \&(0,1,2) & \&(3) & \&(0) & \&(1,2,3) & \&(0,3) & \&(1,2)\\
\hline
& & & & & & & &\\
& & 
\includegraphics[width=0.12\textwidth]{pics/gates_schematic/inverter} & 
$ \overline{NAND} $ &
\includegraphics[width=0.12\textwidth]{pics/gates_schematic/NAND} &
$ \overline{NOR} $ &
\includegraphics[width=0.12\textwidth]{pics/gates_schematic/NOR} & \includegraphics[width=0.12\textwidth]{pics/gates_schematic/XOR} & 
\\
\hline
\#Trans & & 2 & 6 & 4 & 6 & 4 & 8 & \\
\hline
\end{tabular}
\end{sidewaystable}

\end{document}

%\section{FETs}
%	\begin{tabular}{|l|c|p{2cm}|p{2.5cm}|}
%		\hline
%		 & & FET leitend & FET sperrend\\
%		\hline
%		p-FET & \includegraphics[width=0.05\textwidth]{pics/pFET} & G: Low \newline D: High & G: High \newline D: Udef \\
%		\hline
%		n-FET & \includegraphics[width=0.05\textwidth]{pics/nFET} & G: High \newline D: Low & G: Low \newline D: Udef \\
%		\hline
%	\end{tabular}
