%Schriftgr�sse, Layout, Papierformat, Gleichungen Linksb�ndig
\documentclass[10pt,twoside,a4paper,fleqn]{article}
\usepackage[left=1cm,right=1cm,top=1cm,bottom=1cm,includeheadfoot]{geometry}
\usepackage[latin1]{inputenc}
\usepackage[ngerman]{babel,varioref}
%Packages - Von LinAlg Formelsammlung kopiert! �berpr�fen ob die alle notwendig sind
\usepackage{array,amsmath,amssymb,fancybox,graphicx,color,lastpage,wrapfig,fancyhdr,hyperref,verbatim,paralist}
\usepackage{listings}
%Package f�r Bilder im Fliesstext
\usepackage{multicol}

%Package f�r Tabelle im Querformat
\usepackage{rotating}

\raggedright
%Titel, Autor
\newcommand{\titleinfo}{Digitaltechnik DT1 - Zusammenfassung}
\newcommand{\authorinfo}{G. Koeppel, D. Koeppel, L. Leuenberger}
\newcommand{\versioninfo}{v2.0 / Januar 2013}

%pdf info
\hypersetup{pdfauthor={\authorinfo},pdftitle={\titleinfo},colorlinks=false}
%linkbordercolor=white
\author{\authorinfo}
\title{\titleinfo}

%Kopf- und Fusszeile
\pagestyle{fancy}
\fancyhf{}
%Linien oben und unten
\renewcommand{\headrulewidth}{0.5pt} 
\renewcommand{\footrulewidth}{0.5pt}

\fancyhead[L]{\titleinfo{ }\tiny{(\versioninfo)}}
%Kopfzeile rechts bzw. aussen
\fancyhead[R]{Seite \thepage { }von \pageref{LastPage}}
%Fusszeile links bzw. innen
\fancyfoot[L]{\footnotesize{\authorinfo}}
%Fusszeile rechts bzw. ausen
\fancyfoot[R]{\footnotesize{\today}}

%Document Anfang
\begin{document}

\section{Von Analog zu Digital}
	\begin{minipage}[c]{5 cm}
		\includegraphics[width=0.9\textwidth]{pics/von_analog_zu_digital.jpg}
	\end{minipage}
	\begin{minipage}[c]{13 cm}
		\begin{minipage}[c]{4 cm}
			Abtasttheorem:
			\newline\newline\newline
			Amplitudenaufl�sung:
			\newline\newline\newline\newline\newline\newline
			Quantisierungsfehler:
			\newline\newline
			dynamischer Bereich AD-Wandler:
		\end{minipage}
		\begin{minipage}[c]{9 cm}
			Nyquist-Shannon besagt, dass ein Signal mit $f_{max}$ mit mindestens einer Frequenz von $2*f_{max}$ abgetastet werden muss, um das Ursprungssignal wieder herzustellen.\\
			Die Anzahl abz�hlbarer Amplitudenwerte (Quantisierungsstufen) bestimmt die Aufl�sung eines AD-Wandlers. Diese wird in der Regel in Bits angegeben. Je kleiner der Bereich aufgeteilt ist, desto kleiner ist der Abstand $\Delta A$ zwischen zwei benachbarten Amplitudenwerten.\\
			Anz. Quantisierungsstufen: $n_{q} = \frac{V_{max} - V{min}}{\Delta A} =$ n \\
			Bei linearer Quantisierung ist der Quantisierungsfehler (Quantisierungsrauschen) max. $\frac{\Delta A}{2}$\\
			\newline
			$DR_{ADC} = U_{max} / U_{min}$\\
			Faustformel: $DR_{ADC}[dB] = 6 * \# Bit$\\
		\end{minipage}
	\end{minipage}

\section{Bin"are Multiplikation}
	\begin{multicols}{2}
		\includegraphics[width=0.7\columnwidth]{pics/multiplikation}
		\columnbreak
		
		Multiplikation von zwei n-Bit W�rtern\\
		Gr�sstes zu erwartendes Ergebnis:\\
		\ \newline
		$E=(2^n-1)*(2^n-1)=2^{2n}-2^{n+1}+1\leq2^{2n}-1$\\
		\ \newline\newline
		Ergebnis kann also max. 2n Bits lang sein!
	\end{multicols}

\section{CMOS Beschaltung}
	\subsection{Beispiel: Inverter}
		\begin{multicols}{2}
			\includegraphics[width=0.5\textwidth]{pics/cmos_transistorenL.png}
			\columnbreak
			
			\includegraphics[width=0.5\textwidth]{pics/cmos_transistorenR.png}\newline
			PMOS leitet bei Low-Pegel\\
			NMOS leitet bei High-Pegel\\
			\columnbreak
		\end{multicols}
	
\section{RAM}
	\begin{multicols}{2}
		\subsection{SRAM}
		\includegraphics[width=0.5\columnwidth]{pics/sram.png}\newline\newline
		Vorteil: Zustand bleibt, solange Strom vorhanden
			
		\subsection{DRAM}
		\includegraphics[width=0.3\columnwidth]{pics/dram.png}\newline
		Vorteil: Grosse Dichte einzelner Zellen\\
		Nachteile: \\-Leckstrom Kapazit"at\\
				   -Refresh\\
				   -Steuerlogik\\

		
	\end{multicols}
			

		
	

\section{Zahlendarstellungen und Codes}

\subsection{Zahlensysteme}
	\subsubsection{Zahlensysteme ohne festen Stellenwert}
		\begin{compactitem}
			\item R�misches Zahlensystem (keine Null, kein fester Stellenwert, schlechte Unterscheidung)
		\end{compactitem}
		
	\subsubsection{Zahlensysteme mit festem Stellenwert}
		\begin{multicols}{3}
			\begin{compactitem}
				\item Babylon (Basis B = 60)
				\item Maya (Basis B = 20)
			\end{compactitem}
			\columnbreak
			\begin{compactitem}
				\item Dezimal (Basis B = 10)
				\item Bin�r (Basis B = 2)
			\end{compactitem}
			\columnbreak
			\begin{compactitem}
				\item Octal (Basis B = 8)
				\item Hexadezimal (Basis B = 16)
			\end{compactitem}
		\end{multicols}
		
		\begin{minipage}{19 cm}
			Die Wertigkeit des Symbols h�ngt von seiner Position innerhalb der Symbolkette ab:\\
			\newline
			\begin{minipage}[c]{3 cm}
				$z=\sum\limits_{k=0}^{n-1}a_k*B^k$
			\end{minipage}
			\begin{minipage}[c]{6 cm}
				z: Wert der Zahl (im Dezimalsystem)\\
				a: Nennwert der Ziffer\\
				B: Basis des Zahlensystems\\
				n: Stellenanzahl\\
			\end{minipage}
			\begin{minipage}[c]{2.4 cm}
				Bsp: $4156.78=$\\
			\end{minipage}
			\begin{minipage}[c]{6.6 cm}
				$4*10^3+1*10^2+5*10^1+6*10^0$\\
				$+7*10^{-1}+8*10^{-2}$\\
			\end{minipage}
		
			Auch gebrochene Zahlen k�nnen nach dem gleichen Muster bin�r dargestellt werden. Wichtig ist, dass das Komma immer an einer festen Stelle steht (Festkommadarstellung). Definiert ist, dass eine Bin�rzahl 8 Ziffern (n) vor dem Komma und 4 Ziffern (m) nach dem Komma besitzt. Eine gebrochene Bin�rzahl sieht dann so aus:\\
			\newline
			$z_2=a_{n-1}a_{n-2}\dots a_1a_0.a_{-1}a_{-2}\dots a_{-m+1}a_{-m}$\\
			\newline
			Der gesuchte Zahlenwert im Dezimalsystem wird dann folgendermassen berechnet:\\
			\newline
			$z_{10}=a_{n-1}*B^{n-1}+a_{n-2}*B^{n-2}+\dots +a_0*B^0+a_{-1}*B^{-1}+a_{-2}*B^{-2}+\dots +a_{-m+1}*B^{-m+1}+a_{-m}*B^{-m}$
		\end{minipage}

\subsection{Gebr�uchliche polyadische Zahlensysteme}
	\begin{tabular}{|l|l|l|l|l|}
		\hline
			System & Basis & Stellenwerte & Ziffern & Beispiel\\
		\hline
		\hline
			Bin�r & 2 & $\dots$ $2^2$ $2^1$ $2^0$ $\dots$ & 0, 1 & $110_{(2)}=6_{(10)}$\\
		\hline
			Oktal & 8 & $\dots$ $8^2$ $8^1$ $8^0$ $\dots$ & 0, 1, 2, 3, 4, 5, 6, 7 & $273_{(8)}=187_{(10)}$\\
		\hline
			Dezimal & 10 & $\dots$ $10^2$ $10^1$ $10^0$ $\dots$ & 0, 1, 2, 3, 4, 5, 6, 7, 8, 9 & $192_{(10)}=192_{(10)}$\\
		\hline
			Hexadezimal & 16 & $\dots$ $16^2$ $16^1$ $16^0$ $\dots$ & 0, 1, 2, 3, 4, 5, 6, 7, 8, 9, A, B, C, D, E, F & $2$AFF$_{(16)}=11007_{(10)}$\\
		\hline
	\end{tabular}

\subsection{Begriffe im Zusammenhang mit dem bin�ren Zahlensystem}
	\begin{compactitem}	
		\item 
			\begin{tabbing}
				xxxxxxxxxxx\=xxxxxxxxxxxxxxxxxxxxxxxxxxxxxxxxxxxxx\kill	
				Bit (b): \>
							Binary Digit: Kleinsm�gliche Speichereinheit in der Digitaltechnik. Kann zwei m�gliche Zust�nde\\
				 		\>	annehmen: 0 und 1
			\end{tabbing}
		\item 
			\begin{tabbing}
				xxxxxxxxxxx\=xxxxxxxxxxxxxxxxxxxxxxxxxxxxxxxxxxxxx\kill	
				Byte (B): \>
							Einheit von 8 Bits. Auch genannt Oktett: 1 Oktett = 1 Byte = 8 Bit. Byte ist die Standartbezeichnung\\
						\>	von Speicherkapazit�ten und Datenmengen.
			\end{tabbing}
		\item 
			\begin{tabbing}
				xxxxxxxxxxx\=xxxxxxxxxxxxxxxxxxxxxxxxxxxxxxxxxxxxx\kill	
				Nibble: \>
							Bin�rzahlen in Gruppen von 4 Bits
			\end{tabbing}
		\item 
			\begin{tabbing}
				xxxxxxxxxxx\=xxxxxxxxxxxxxxxxxxxxxxxxxxxxxxxxxxxxx\kill	
				MSB: \>
							Most Significant Bit. Bit mit h�chster Wertigkeit, steht ganz links im bin�ren Wort
			\end{tabbing}
		\item 
			\begin{tabbing}
				xxxxxxxxxxx\=xxxxxxxxxxxxxxxxxxxxxxxxxxxxxxxxxxxxx\kill	
				LSB: \>
							Least Significant Bit. Bit mit tiefster Wertigkeit, steht ganz rechts im bin�ren Wort
			\end{tabbing}
	\end{compactitem}
	
\subsection{Umwandlung von Dezimalzahlen}
	Beispiel f�r die Umwandlung der Zahl $109.78125_{(10)}$:
	\begin{tabular}{|lllllll|lllllll|lllllll|}
		\hline
			\multicolumn{7}{|l|}{Dezimal zu Bin�r} & \multicolumn{7}{|l|}{Dezimal zu Oktal} & \multicolumn{7}{|l|}{Dezimal zu Hexadeximal}\\
		\hline
		\hline
			109 & : & 2 & = & 54 & Rest: & 1 &
			109 & : & 8 & = & 13 & Rest: & 5 & 
			109 & : & 16 & = & 6 & Rest: & D \\
			
			54 & : & 2 & = & 27 & Rest: & 0 &
			13 & : & 8 & = & 1 & Rest: & 5 & 
			6 & : & 16 & = & 0 & Rest: & 6 \\
			
			27 & : & 2 & = & 13 & Rest: & 1 &
			1 & : & 8 & = & 0 & Rest: & 1 & 
			\multicolumn{7}{|l|}{} \\
						
			13 & : & 2 & = & 6 & Rest: & 1 &
			\multicolumn{7}{|l|}{} & 
			\multicolumn{7}{|l|}{} \\
			
			6 & : & 2 & = & 3 & Rest: & 0 &
			\multicolumn{7}{|l|}{} & 
			\multicolumn{7}{|l|}{} \\
			
			3 & : & 2 & = & 1 & Rest: & 1 &
			\multicolumn{7}{|l|}{} & 
			\multicolumn{7}{|l|}{} \\
			
			1 & : & 2 & = & 0 & Rest: & 1 &
			\multicolumn{7}{|l|}{} & 
			\multicolumn{7}{|l|}{} \\
		\hline
			0.78125 & * & 2 & = & 0.5625 & + & 1 &
			0.78125 & * & 8 & = & 0.25 & + & 6 &
			0.78125 & * & 16 & = & 0.5 & + & C \\
			
			0.5625 & * & 2 & = & 0.125 & + & 1 &
			0.25 & * & 8 & = & 0 & + & 2 &
			0.5 & * & 16 & = & 0 & + & 8 \\
			
			0.125 & * & 2 & = & 0.25 & + & 0 &
			\multicolumn{7}{|l|}{} &
			\multicolumn{7}{|l|}{} \\
			
			0.25 & * & 2 & = & 0.5 & + & 0 &
			\multicolumn{7}{|l|}{} &
			\multicolumn{7}{|l|}{} \\
			
			0.5 & * & 2 & = & 0 & + & 1 &
			\multicolumn{7}{|l|}{} &
			\multicolumn{7}{|l|}{} \\
		\hline
		\hline
			\multicolumn{7}{|l|}{$109_{(10)}=1101101.11001_{(2)}$} & \multicolumn{7}{|l|}{$109_{(10)}=155.62_{(8)}$} & \multicolumn{7}{|l|}{$109_{(10)}=6$D$.$C$8_{(16)}$}\\
		\hline
	\end{tabular}
	
	
	
\section{Schaltalgebra}
\subsection{Symbolik}
	Folgende Symbole werden verwendet:\\
	\begin{multicols}{4}
		$\neg=^-=$NOT
	\columnbreak
	
		$\vee=+=$OR
	\columnbreak
	
		$\wedge=*=$AND
	\columnbreak
	
		$\oplus=\$=$(E)XOR
	\end{multicols}
	
\subsection{Rechenregeln}
	\begin{tabular}{llll}
		Verkn"upfung mit 0 & $ a \vee 0 = a $ & $ a \wedge 0 = 0 $ & $ a \oplus 0 = a $\\
		Verkn"upfung mit 1 & $ a \vee 1 = 1 $ & $ a \wedge 1 = a $ & $ a \oplus 1 = \overline{a} $ \\
		Verkn. mit sich selbst & $ a \vee a = a $ & $ a \wedge a = a $ & $ a \oplus a = 0 $ \\
		Verkn. mit Inversem & $ a \vee \overline{a} = 1 $ & $ a \wedge \overline{a} = 0 $ & $ a \oplus \overline{a} = 1 $ \\
		\\
		Kommutativgesetz & $ a \vee b = b \vee a $ & $ a \wedge b = b \wedge a $ & $ a \oplus b = b \oplus a $\\
		Assioziativgesetz & $ (a \vee b) \vee c = a \vee (b \vee c) $ & $ (a \wedge b) \wedge c = a \wedge (b \wedge c) $ & $ (a \oplus b) \oplus c = a \oplus (b \oplus c) $ \\
		Distributivgesetz & $ a \wedge (b \vee c) = (a \wedge b) \vee (a \wedge c) $ & $ a \vee (b \wedge c) = (a \vee b) \wedge (a \vee c) $ & $ a \wedge (b \oplus c) = (a \wedge b) \oplus (a \wedge c) $ \\	
		\end{tabular}
		
\subsubsection{Vereinfachungen \& Minimierungen}
	\begin{multicols}{4}
		$ a \vee (a \wedge b) = a $ \\
		$ a \wedge (a \vee b) = a $ \\
		$ (a \vee b) \wedge (a \vee \neg b) = a $ 
	\columnbreak
	
		$ (a \wedge \overline{b}) \vee b = a \vee b $ \\
		$ (a \vee \overline{b}) \wedge b = a \wedge b $ \\
		$ (a \wedge b) \vee (a \wedge \neg b) = a $ 
	\columnbreak
	
		$ (a \wedge \overline{b}) \oplus b = a \vee b $ \\
		$ (a \oplus \overline{b}) \wedge b = a \wedge b $ \\
		$ (a * b) + (a * \neg b) = a $ 
	\columnbreak
		
		$ (a \wedge b) \vee (a \wedge \overline{b}) = a $\\	
		$ (a \vee b) \wedge (a \vee \overline{b}) = a $ \\
		$ (a + b) * (a + \neg b) = a $ 
	\end{multicols}

%\subsection{Minimierungsregeln}
%\begin{multicols}{4}
%	$ (a \vee b) \wedge (a \vee \neg b) = a $ \\
%	\columnbreak
%	$ (a \wedge b) \vee (a \wedge \neg b) = a $ \\
%	\columnbreak
%	$ (a * b) + (a * \neg b) = a $ \\
%\columnbreak
%	$ (a + b) * (a + \neg b) = a $ \\
%\end{multicols}


\subsection{Shannon und DeMorgan}
	\begin{tabular}{lll}
		Ursprungsschaltung: & Shannon & DeMorgan\\
		\includegraphics[width=0.3\textwidth]{pics/shanonursprung} & 
		\includegraphics[width=0.3\textwidth]{pics/demorganende} &
		\includegraphics[width=0.3\textwidth]{pics/shanonende}\\
		Shannon & DeMorgan & \\
		$\neg f(a, b, c, ...z; \wedge, \vee) = f(\neg a, \neg v, \neg c, ... \neg z; \wedge, \vee)$ & $\neg(a \vee b) = \neg a \wedge \neg b$ & 
		  $\neg(a \wedge b) = \neg a \vee \neg b $ \\ 
	\end{tabular}
% Die Negation einer beliebigen log. Funktion f erh"alt man, wenn man alle Variablen durch ihre Negation ersetzt und die Operatoren vertauscht. \\

\subsection{Normalformen: KDNF und KKNF}
	\begin{minipage}{10cm}
		Ausgangslage:\\
		\begin{tabular}{|l|l|l|l|l||l||l|}
			\hline	
				Dezimal & $x_2$ & $x_1$ & $x_0$ & y & KDNF & KKNF\\	
			\hline
			\hline
				0 & 0 & 0 & 0 & 0 & & $x_2 \vee x_1 \vee x_0$ \\
			\hline	
				1 & 0 & 0 & 1 & 1 &	$\overline{x_2} \wedge \overline{x_1} \wedge x_0$ & \\
			\hline
				2 & 0 & 1 & 0 & 0 & & $x_2 \vee \overline{x_1} \vee x_0$  \\
			\hline
				3 & 0 & 1 & 1 & 1 &	$\overline{x_2} \wedge x_1 \wedge x_0$ & \\
			\hline
				4 & 1 & 0 & 0 & 0 & & $\overline{x_2} \vee x_1 \vee x_0$ \\
			\hline
				5 & 1 & 0 & 1 & d & $[x_2 \wedge \overline{x_1} \wedge x_0]$ & $[\overline{x_2} \vee x_1 \vee \overline{x_0}]$  \\
			\hline
				6 & 1 & 1 & 0 & 1 & $x_2 \wedge x_1 \wedge \overline{x_0}$ &  \\
			\hline
				7 & 1 & 1 & 1 & 1 & $x_2 \wedge x_1 \wedge x_0$ &\\
			\hline
		\end{tabular}
\end{minipage}
\begin{minipage}{8cm}
Kanonisch disjunktive Normalform \lbrack KDNF\rbrack: Es werden nur diejenigen Zeilen der Wahrheitstabelle aufgef"uhrt, deren Funktionswert 1 oder d ist. \\
$y=(\overline{x_2} \wedge \overline{x_1} \wedge x_0) \vee (\overline{x_2} \wedge x_1 \wedge x_0) \vee [x_2 \wedge \overline{x_1} \wedge x_0] \vee (x_2 \wedge x_1 \wedge \overline{x_0}) \vee (x_2 \wedge x_1 \wedge x_0)$ \\
\\
Kanonisch konjunktive Normalform \lbrack KKNF\rbrack: 
Es werden nur diejenigen Zeilen der Wahrheitstabelle aufgef"uhrt, deren Funktionswert 0 oder d ist. \\
$y=(x_2 \vee x_1 \vee x_0) \wedge (x_2 \vee \overline{x_1} \vee x_0) \wedge (\overline{x_2} \vee x_1 \vee x_0) \wedge [\overline{x_2} \vee x_1 \vee \overline{x_0}]$ \\
\end{minipage}
\subsection{Karnaugh-Diagramm}
	\subsubsection{Arbeiten mit dem KV-Diagramm}
\begin{compactitem}
	\item 1) \ \ Aufstellen der Wahrheitstabelle.\\
	\item 2) \ \ "Ubertragen der Werte der Wahrheitstabelle ins KV Diagramm.\\
	\item 3) \ \ M"oglichst grosse Gruppen a $2^n$ Felder bilden (k"onnen auch "uber die Ecken gehen).\\
	\item 4)
	\begin{tabular}{ll}
		Kanonisch disjunktive Normalform: & Kanonisch konjunktive Normalform: \\
		Gruppen von Feldern mit Wert 1 oder d & Gruppen von Feldern mit Wert 0 oder d\\
		Primimplikanten: AND Verkn"upfung & Primimplikanten: OR Verkn"upfung\\
		OR Verkn"upfung aller Primimplikanten & AND Verkn"upfung aller Primimplikanten\\
	\end{tabular}
\end{compactitem}
\newpage
\subsection{Karnaugh-Diagramm}
	\begin{multicols}{3}
		\includegraphics[width=0.3\textwidth]{pics/kv/2erKV} 
	\columnbreak
	
		\includegraphics[width=0.3\textwidth]{pics/kv/3erKV}
	\columnbreak
	
		\includegraphics[width=0.3\textwidth]{pics/kv/4erKV}
	\end{multicols}
	
%	\subsubsection{Arbeiten mit dem KV-Diagramm}
%		\begin{compactitem}
%			\item 1) \ \ Aufstellen der Wahrheitstabelle.\\
%			\item 2) \ \ "Ubertragen der Werte der Wahrheitstabelle in KV Diagramm.\\
%			\item 3) \ \ M"oglichst grosse Gruppen a $2^n$ Felder bilden.\\
%			\item 4)
%				\begin{tabular}{ll}
%					Kanonisch disjunktive Normalform: & Kanonisch konjunktive Normalform: \\
%					Gruppen von Feldern mit Wert 1 oder d & Gruppen von Feldern mit Wert 0 oder d\\
%					Primimplikanten: AND Verkn"upfung & Primimplikanten: OR Verkn"upfung\\
%					OR Verkn"upfung aller Primimplikanten & AND Verkn"upfung aller Primimplikanten\\
%				\end{tabular}
%		\end{compactitem}
\section{Logische Gatter}

\subsubsection{Ralisierungsaufwand}
Anzahl Transistoren ist ein m"ogliches Mass f"ur Komplexit"at (Marketing). In der Technik spricht man von Gatter"aquivalent: 1 Gatter"aquivalent entspricht 4 Transistoren.

\subsection{Anzahl Schaltfunktionen}
Permutation: $F= {2^2}^n$ F: Anzahl m"oglicher Schaltfunktionen, n: Anzahl Eing"ange

	\subsection{Verhalten logischer Gatter}
		\begin{multicols}{2}
			\subsubsection{St"orabstand}
				High-Pegel: $ U_{nH} = U_{aHmin} - U_{eHmin} $\\
				Low-Pegel: $ U_{nL} = U_{eLmax} - U_{aLmax} $\\
				
		%\end{multicols}

		%\begin{multicols}{2}
			\subsubsection{Propagation delay Time Anstiegs- bwz. Abstiegszeit}
				Zeit zwischen 50\% von $V_{max}$ am Eingang und 50\% von $V_{max}$ am Ausgang.\\
				$t_{pd}=\frac{t_{pLH}+t{pHL}}{2}$\\
				
			\subsubsection{Transition delay Time (Verz"ogerungszeit)}
				Zeit zwischen 10\% und 90\% von $V_{max}$.\\
				$t_{tLH}$: Transition time low to high.\\
				$t_{tHL}$: Transition time high to low.\\
				\includegraphics[width=0.2\textwidth]{pics/delay}
				\includegraphics[width=0.2\textwidth]{pics/Pegelbereiche_Stoerabstand}
		\end{multicols}
		
		
\subsection{Aufbau logischer Gatter}

%\newpage
\begin{sidewaystable}
\begin{tabular}{|c|c|c|c|c|c|c|c|c|}
\hline
Funktion & Buffer & NOT & AND & NAND & OR & NOR & EXOR & XNOR\\
& & Nicht & Und & Nicht Und & Oder & Nicht Oder & Exklusiv Oder & Nicht Ex. Oder\\
& & Inverter & Konjunktion & & Disjunktion & & Antivalenz & "Aquivalenz \\
\hline
Formel & a & $ \overline a $ & $ a \cdot b $ & $ \overline{a \cdot b} $ & $ a + b $ & $ \overline{a + b} $ & $ a \oplus b $ & $ \overline{a \oplus b} $\\
& a & $ \overline a $ & $ a \wedge b $ & $ \overline{a \wedge b} $ & $ a \vee b $ & $ \overline{a \vee b} $ & $ a \not= b $ & $ \overline{a \not= b} $ \\
& a & !a & $ a \& b $ & $ !(a \& b) $ & a\#b & !(a\#b) & a\$b & !(a\$b) \\
& & & & & & & $ a \veebar b $ & $ \overline{a \veebar b} $\\
\hline
& & & & & & & &\\
& \includegraphics[width=0.08\textwidth]{pics/gates_symbol/buffer} & \includegraphics[width=0.08\textwidth]{pics/gates_symbol/not} & \includegraphics[width=0.08\textwidth]{pics/gates_symbol/and} & \includegraphics[width=0.08\textwidth]{pics/gates_symbol/nand} & \includegraphics[width=0.08\textwidth]{pics/gates_symbol/or} & \includegraphics[width=0.08\textwidth]{pics/gates_symbol/nor} & \includegraphics[width=0.08\textwidth]{pics/gates_symbol/exor} & \includegraphics[width=0.08\textwidth]{pics/gates_symbol/xnor} \\
\hline
(0,0) & 0 & 1 & 0 & 1 & 0 & 1 & 0 & 1\\
(0,1) &   &   & 0 & 1 & 1 & 0 & 1 & 0\\
(1,0) & 1 & 0 & 0 & 1 & 1 & 0 & 1 & 0\\
(1,1) &   &   & 1 & 0 & 1 & 0 & 0 & 1\\
\hline
KDNF & \#(1) & \#(0) & \#(3) & \#(0,1,2) & \#(1,2,3) & \#(0) & \#(1,2) & \#(0,3) \\
KKNF & \&(0) & \&(1) & \&(0,1,2) & \&(3) & \&(0) & \&(1,2,3) & \&(0,3) & \&(1,2)\\
\hline
& & & & & & & &\\
& & 
\includegraphics[width=0.12\textwidth]{pics/gates_schematic/inverter} & 
$ \overline{NAND} $ &
\includegraphics[width=0.12\textwidth]{pics/gates_schematic/NAND} &
$ \overline{NOR} $ &
\includegraphics[width=0.12\textwidth]{pics/gates_schematic/NOR} & \includegraphics[width=0.12\textwidth]{pics/gates_schematic/XOR} & 
\\
\hline
\#Trans & & 2 & 6 & 4 & 6 & 4 & 8 & \\
\hline
\end{tabular}
\end{sidewaystable}
\section{Sequentielle Systeme}

 \begin{multicols}{2}
	\subsection{Taktsignal}
			\includegraphics[width=0.6\columnwidth]{pics/taktsignal}
			\newline
			$\mathbf{f=\frac{1}{T}}$ [Hz]
			\columnbreak
		\subsection{l"angster Pfad}
			\includegraphics[width=0.5\columnwidth]{pics/laengster_pfad.jpg}
			\columnbreak
	\end{multicols}

\subsection{Flipflops und Latches}
	\subsubsection{Unterschied Flipflop und Latch}
%		\begin{minipage}{12 cm}
%			Taktzustandsgesteuerte Systeme haben den Nachteil, dass in ihrer transparenten Phase auch asynchrone Schaltvorg"ange stattfinden k"onnen. Echt synchrone Systeme "andern ihren Zustand nur bei der aktiven Flanke des Taktsignals. Genau in diesem Moment und sonst nie wird das Eingangssignal bei einem Speicherelement in den Speicher "ubertragen. Nur beim taktflankengesteuerten System wechseln die Ausg"ange immer genau zum Zeitpunkt der aktiven Taktflanke. Beim taktzustandsgesteuerten System sind w"ahrend der transparenten Phase auch Zustands"anderungen zwischen zwei Taktflanken m"oglich. \\
%			Taktflankengesteuerte Speicherelemente werden Flip-Flops genannt. Taktzustandsgesteuerte Speicherelemente werden Latches genannt
%		\end{minipage}
%		\begin{minipage}{0.5 cm}
%			\ 
%		\end{minipage}
%		\begin{minipage}{6 cm}
%			\includegraphics[width=0.9\textwidth]{pics/flipflop_latch}
%		\end{minipage}
		\begin{multicols}{3}
			\textbf{D-FF}\\
			\includegraphics[width=0.2\columnwidth]{pics/d_flipflopO.png}
			\includegraphics[width=0.78\columnwidth]{pics/d_flipflop.png}\\
			%Wechselt Zustand, nur bei (positiver Flanke des Taktsignals)
			\columnbreak
			
			\textbf{D-Latch}\\
			\includegraphics[width=0.2\columnwidth]{pics/d_latchO.png}
			\includegraphics[width=0.78\columnwidth]{pics/d_latch.png}\\
			Wechselt Zustand, wann immer Level der Eingangssignale wechseln
			\columnbreak
			
			\textbf{Unterschied}\\
			\includegraphics[width=0.8\columnwidth]{pics/flipflop_latch}
			\columnbreak
			
		\end{multicols}
			
		
	\subsubsection{Transmission Gate}
		\begin{minipage}{10 cm}
			Das Transmission Gate ist von seiner Funktion her ein einfacher Schalter, der Signale sowohl auf positivem, als auch auf negativem Pegel schalten kann.
		\end{minipage}
		\begin{minipage}{0.5 cm}
			\ 
		\end{minipage}
		\begin{minipage}{8 cm}
			\includegraphics[width=0.9\textwidth]{pics/transmissiongate}
		\end{minipage}
		
	\begin{multicols}{2}
		\subsubsection{RS-Latch}
			\begin{minipage}{4 cm}
				\includegraphics[width=0.9\textwidth]{pics/rs_latch}
			\end{minipage}
			\begin{minipage}{4 cm}
				\begin{tabular}{|cc|cc|}
					\hline
						S & R & $Q$ & $\overline{Q}$ \\
					\hline	
						0 & 0 & $Q$ & $\overline{Q}$ \\
						0 & 1 & 0 & 1 \\
						1 & 0 & 1 & 0 \\
						1 & 1 & \multicolumn{2}{c|}{nicht definiert} \\
					\hline
				\end{tabular}
			\end{minipage}
		
		\subsubsection{RS-Latch mit Clock}
			\includegraphics[width=0.4\textwidth]{pics/rs_latch_clock}
		\columnbreak
		
		\subsubsection{D-Latch}
			\begin{minipage}{4 cm}
				\includegraphics[width=0.9\textwidth]{pics/dlatch}
			\end{minipage}
			\begin{minipage}{4 cm}
				\begin{tabular}{|cc|cc|}
					\hline
						D & C & $Q$ & $\overline{Q}$ \\
					\hline	
						0 & 0 & $Q$ & $\overline{Q}$ \\
						0 & 1 & 0 & 1 \\
						1 & 0 & $Q$ & $\overline{Q}$ \\
						1 & 1 & 0 & 1 \\
					\hline
				\end{tabular}
			\end{minipage}
			
		\subsubsection{D-Flipflop mit Reset}
			\includegraphics[width=0.4\textwidth]{pics/dflipflop}
	\end{multicols}
	
	\subsubsection{Setup- und Holdtime}
		\begin{minipage}{10 cm}
			\begin{compactitem}
				\item $t_s$= setup time $\rightarrow$ Minimale Zeitspanne, w"ahrend der ein Datensignal vor einer aktiven Clockflanke stabil sein muss, um zuverl"assig eingelesen zu werden.
				\item $t_H$= hold time $\rightarrow$ Minimale Zeitspanne, w"ahrend der ein Datensignal nach einer aktiven Clockflanke noch stabil bleiben muss, damit der Einlesevorgang des Datensignals erfolgreich abgeschlossen werden kann.
			\end{compactitem}
		\end{minipage}
		\begin{minipage}{0.5 cm}
			\ 
		\end{minipage}
		\begin{minipage}{8 cm}
			\includegraphics[width=0.9\textwidth]{pics/setupholdtime}
		\end{minipage}
	\newpage
		
\subsection{Beschreibung sequentieller Systeme}
	\begin{multicols}{2}
		\begin{compactitem}
			\item S: Menge der Zust"ande mit Zustandsaktionen
			\item I$\subseteq$S: Initalzust"ande
			\item T: Kombinatorische "Ubergangsrelation
			\item E: Eingangssignale
			\item A: Ausgangssignale
		\end{compactitem}
	\end{multicols}
	
	\begin{multicols}{2}
		\subsubsection{Tabellarische Beschreibung}
			\includegraphics[width=0.49\textwidth]{pics/zustandstabelle}
		\columnbreak
		
		\subsubsection{Grafische Beschreibung (Zustandsdiagramm)}
			\includegraphics[width=0.5\textwidth]{pics/zustandsdiagramm}
	\end{multicols}
	Diese Beispiele visualisieren das sequentielle System des Watch-Controllers.
	
\subsection{Strukturen der Finite State Machine}
	\begin{multicols}{3}
		\subsubsection{Grundstruktur}
			\includegraphics[width=0.3\textwidth]{pics/seq_grundstruktur}
		\columnbreak
		
			\begin{compactitem}
				\item s: Zustand, Zustandsvektor
				\item x: Prim"are Eing"ange, Eingangsvektor
				\item y: Prim"are Ausg"ange, Ausgangsvektor
				\item d: Speicheransteuerung, Folgezustand
				\item m: Anzahl Eing"ange
				\item n: Anzahl Ausg"ange
				\item k: Anzahl Speicherstellen
				\item F: Funktion f"ur die Ausg"ange
				\item G: Funktion f"ur die Speicheransteuerung
				\item Z: Zustandsspeicher
				\item I: Index der aktuellen Taktflanke (Taktflankennummer)
			\end{compactitem}
	\end{multicols}
	
	\begin{multicols}{3}
		\subsubsection{Mealy-System}
			\includegraphics[width=0.26\textwidth]{pics/seq_mealy}
			Ausg"ange h"angen vom momentanen Zustand und den aktuellen Eing"angen ab.
			\columnbreak
			
		\subsubsection{Moore-System}
			\includegraphics[width=0.26\textwidth]{pics/seq_moore}
			Ausg"ange h"angen nur vom momentanen Zustand ab und "andern mit der Clock Flanke.
			\columnbreak
			
		\subsubsection{Medwedjew-System}
			\includegraphics[width=0.3\textwidth]{pics/seq_medmedjew}
			Die prim"aren Ausg"ange entsprechen dem Zustandsvektor.
			\columnbreak
	\end{multicols}
	
\subsection{Zustandscodierung (ZC)}
	k: Anzahl Speicherstellen (1 Bit ist eine Speicherstelle), p: Anzahl Zust"ande
	\begin{multicols}{3}
		\begin{compactitem}
			\item Bin"ar: Alle Zust"ande werden der Reihe nach durchnummeriert. \\
			$k = \lceil \frac{log_{10}(p)}{log_{10}(2)} \rceil = \lceil log_2(p)\rceil, \; p < 2^k $
			Anz. m"ogl. ZC: $q = (2^k!)/(2^k - p)!$
			\item ONE-HOT: Nur eine Speicherstelle im Code hat jeweils den Wert 1. Alle anderen besitzen den Wert 0 (z.B. 001, 010, 100)
			\item ONE-COLD: Nur eine Speicherstelle im Code hat jeweils den Wert 0. Alle anderen besitzen den Wert 1 (z.B. 110, 101, 011)
		\end{compactitem}
	\end{multicols}

%\subsection{Synthese von Zustandsmaschinen}
%	\begin{multicols}{2}
%		\begin{enumerate}
%			\setlength{\itemsep}{1pt}
%			\setlength{\parskip}{0pt}
%			\setlength{\parsep}{0pt}
%			\item Zustandsdiagramm aufstellen
%			\item Zustandskodierung zuweisen
%			\item Zustandstabelle nach festen Regeln aufstellen
%			\item Speicheransteuer-Funktionen bestimmen
%			\item Ausgangs-Funktionen bestimmen
%		\end{enumerate}
%	\end{multicols}

\end{document}

