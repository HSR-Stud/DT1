\section{Zahlensysteme}

%	\subsubsection{Zahlensysteme ohne festen Stellenwert}
%		\begin{compactitem}
%			\item R�misches Zahlensystem (keine Null, kein fester Stellenwert, schlechte Unterscheidung)
%		\end{compactitem}
%		
%	\subsubsection{Zahlensysteme mit festem Stellenwert}
%		\begin{multicols}{3}
%			\begin{compactitem}
%				\item Babylon (Basis B = 60)
%				\item Maya (Basis B = 20)
%			\end{compactitem}
%			\columnbreak
%			\begin{compactitem}
%				\item Dezimal (Basis B = 10)
%				\item Bin�r (Basis B = 2)
%			\end{compactitem}
%			\columnbreak
%			\begin{compactitem}
%				\item Octal (Basis B = 8)
%				\item Hexadezimal (Basis B = 16)
%			\end{compactitem}
%		\end{multicols}

\subsection{Gebr�uchliche polyadische Zahlensysteme}
	\begin{tabular}{|l|l|l|l|l|}
		\hline
		System & Basis & Stellenwerte & Ziffern & Beispiel\\
		\hline
		\hline
		Bin�r & 2 & $\dots$ $2^2$ $2^1$ $2^0$ $\dots$ & 0, 1 & $110_{(2)}=6_{(10)}$\\
		\hline
		Oktal & 8 & $\dots$ $8^2$ $8^1$ $8^0$ $\dots$ & 0, 1, 2, 3, 4, 5, 6, 7 & $273_{(8)}=187_{(10)}$\\
		\hline
		Dezimal & 10 & $\dots$ $10^2$ $10^1$ $10^0$ $\dots$ & 0, 1, 2, 3, 4, 5, 6, 7, 8, 9 & $192_{(10)}=192_{(10)}$\\
		\hline
		Hexadezimal & 16 & $\dots$ $16^2$ $16^1$ $16^0$ $\dots$ & 0, 1, 2, 3, 4, 5, 6, 7, 8, 9, A, B, C, D, E, F & $2$AFF$_{(16)}=11007_{(10)}$\\
		\hline
	\end{tabular}

\subsection{Bin�r, Oktal, Hexadezimal in Dezimal umwandeln}
		\begin{minipage}{19 cm}
			Die Wertigkeit des Symbols h�ngt von seiner Position innerhalb der Symbolkette ab:\\
			\newline
			\begin{minipage}[c]{3 cm}
				$z=\sum\limits_{k=0}^{n-1}a_k*B^k$
			\end{minipage}
			\begin{minipage}[c]{6 cm}
				z: Wert der Zahl (im Dezimalsystem)\\
				a: Nennwert der Ziffer\\
				B: Basis des Zahlensystems\\
				n: Stellenanzahl\\
			\end{minipage}
			\begin{minipage}[c]{2.4 cm}
				Bsp: $4156.78=$\\
				\newline
			\end{minipage}
			\begin{minipage}[c]{6.6 cm}
				$4*10^3+1*10^2+5*10^1+6*10^0$\\
				$+7*10^{-1}+8*10^{-2}$\\
			\end{minipage}
		
			Auch gebrochene Zahlen k�nnen nach dem gleichen Muster bin�r dargestellt werden. Wichtig ist, dass das Komma immer an einer festen Stelle steht (Festkommadarstellung). Definiert ist, dass eine Bin�rzahl 8 Ziffern (n) vor dem Komma und 4 Ziffern (m) nach dem Komma besitzt. Eine gebrochene Bin�rzahl sieht dann so aus:\\
			\newline
			gebrochene Bin�rzahl:\quad $z_2=a_{n-1}a_{n-2}\dots a_1a_0$ . $a_{-1}a_{-2}\dots a_{-m+1}a_{-m}$\\	
		\end{minipage}

\subsubsection{Beispiel}
	\begin{minipage}{19 cm}
		$z_{10}=a_{n-1}*B^{n-1}+a_{n-2}*B^{n-2}+\dots+a_0*B^0+a_{-1}*B^{-1}+a_{-2}*B^{-2}+\dots+a_{-m+1}*B^{-m+1}+a_{-m}*B^{-m}$\\
		\newline
		
			\begin{tabular}{|l|l|l|}
			\hline
			Bin�r zu Dezimal & Oktal zu Dezimal & Hexadezimal zu Dezimal\\
			\hline
			\hline
			$10.1_{2} = 1*2^1 + 0*2^0 + 1*2^{-1} = 2.5_{10}$ & $7.4_{8} = 7*8^1 + 4*8^{-1} = 56.5_{10}$& $16.4_{16} = 1*16^1 + 6*16^0 + 4*16^{-1} = 22.25_{10}$\\
			\hline
		\end{tabular}
	\end{minipage}

%\subsection{Begriffe im Zusammenhang mit dem bin�ren Zahlensystem}
%	\begin{compactitem}	
%		\item 
%			\begin{tabbing}
%				xxxxxxxxxxx\=xxxxxxxxxxxxxxxxxxxxxxxxxxxxxxxxxxxxx\kill	
%				Bit (b): \>
%							Binary Digit: Kleinsm�gliche Speichereinheit in der Digitaltechnik. Kann zwei m�gliche Zust�nde\\
%				 		\>	annehmen: 0 und 1
%			\end{tabbing}
%		\item 
%			\begin{tabbing}
%				xxxxxxxxxxx\=xxxxxxxxxxxxxxxxxxxxxxxxxxxxxxxxxxxxx\kill	
%				Byte (B): \>
%							Einheit von 8 Bits. Auch genannt Oktett: 1 Oktett = 1 Byte = 8 Bit. Byte ist die Standartbezeichnung\\
%						\>	von Speicherkapazit�ten und Datenmengen.
%			\end{tabbing}
%		\item 
%			\begin{tabbing}
%				xxxxxxxxxxx\=xxxxxxxxxxxxxxxxxxxxxxxxxxxxxxxxxxxxx\kill	
%				Nibble: \>
%							Bin�rzahlen in Gruppen von 4 Bits
%			\end{tabbing}
%		\item 
%			\begin{tabbing}
%				xxxxxxxxxxx\=xxxxxxxxxxxxxxxxxxxxxxxxxxxxxxxxxxxxx\kill	
%				MSB: \>
%							Most Significant Bit. Bit mit h�chster Wertigkeit, steht ganz links im bin�ren Wort
%			\end{tabbing}
%		\item 
%			\begin{tabbing}
%				xxxxxxxxxxx\=xxxxxxxxxxxxxxxxxxxxxxxxxxxxxxxxxxxxx\kill	
%				LSB: \>
%							Least Significant Bit. Bit mit tiefster Wertigkeit, steht ganz rechts im bin�ren Wort
%			\end{tabbing}
%	\end{compactitem}
	
\subsection{Dezimalzahl in Bin�r, Oktal, Hexadezimal umwandeln}
	Beispiel f�r die Umwandlung der Zahl: $109.78125_{(10)}$:\\
	\begin{tabular}{|lllllll|lllllll|lllllll|}
		\hline
			\multicolumn{7}{|l|}{Dezimal zu Bin�r} & \multicolumn{7}{|l|}{Dezimal zu Oktal} & \multicolumn{7}{|l|}{Dezimal zu Hexadeximal}\\
		\hline
		\hline
			109 & : & 2 & = & 54 & Rest: & 1 &
			109 & : & 8 & = & 13 & Rest: & 5 & 
			109 & : & 16 & = & 6 & Rest: & D \\
			
			54 & : & 2 & = & 27 & Rest: & 0 &
			13 & : & 8 & = & 1 & Rest: & 5 & 
			6 & : & 16 & = & 0 & Rest: & 6 \\
			
			27 & : & 2 & = & 13 & Rest: & 1 &
			1 & : & 8 & = & 0 & Rest: & 1 & 
			\multicolumn{7}{|l|}{} \\
						
			13 & : & 2 & = & 6 & Rest: & 1 &
			\multicolumn{7}{|l|}{} & 
			\multicolumn{7}{|l|}{} \\
			
			6 & : & 2 & = & 3 & Rest: & 0 &
			\multicolumn{7}{|l|}{} & 
			\multicolumn{7}{|l|}{} \\
			
			3 & : & 2 & = & 1 & Rest: & 1 &
			\multicolumn{7}{|l|}{} & 
			\multicolumn{7}{|l|}{} \\
			
			1 & : & 2 & = & 0 & Rest: & 1 &
			\multicolumn{7}{|l|}{} & 
			\multicolumn{7}{|l|}{} \\
		\hline
			0.78125 & * & 2 & = & 0.5625 & + & 1 &
			0.78125 & * & 8 & = & 0.25 & + & 6 &
			0.78125 & * & 16 & = & 0.5 & + & C \\
			
			0.5625 & * & 2 & = & 0.125 & + & 1 &
			0.25 & * & 8 & = & 0 & + & 2 &
			0.5 & * & 16 & = & 0 & + & 8 \\
			
			0.125 & * & 2 & = & 0.25 & + & 0 &
			\multicolumn{7}{|l|}{} &
			\multicolumn{7}{|l|}{} \\
			
			0.25 & * & 2 & = & 0.5 & + & 0 &
			\multicolumn{7}{|l|}{} &
			\multicolumn{7}{|l|}{} \\
			
			0.5 & * & 2 & = & 0 & + & 1 &
			\multicolumn{7}{|l|}{} &
			\multicolumn{7}{|l|}{} \\
		\hline
		\hline
			\multicolumn{7}{|l|}{$109_{(10)}=1101101.11001_{(2)}$} & \multicolumn{7}{|l|}{$109_{(10)}=155.62_{(8)}$} & \multicolumn{7}{|l|}{$109_{(10)}=6$D$.$C$8_{(16)}$}\\
		\hline
	\end{tabular}

\subsection{Bin�r in Oktal, Hexadezimal umwandeln}
Vorgehen: Bin�re Zahl in Gruppen aufteilen; Bin�r zu Oktal: 3er-Gruppen; Bin�r zu Hexadezimal: 4er-Gruppen  \\
Bei Komma: Zahl vor dem Komma links mit Nullen auff�llen, Zahl nach Komma rechts mit Nullen auff�llen
	\begin{multicols}{3}
		\begin{tabular}{|l|l|l|l|}
			\hline
			Dezimal & Bin�r & Oktal & Hexadeximal\\
			\hline
			\hline
			0 & 0000 & 0 & 0\\
			\hline
			1 & 0001 & 1 & 1\\
			\hline
			2 & 0010 & 2 & 2\\
			\hline
			3 & 0011 & 3 & 3\\
			\hline
			4 & 0100 & 4 & 4\\
			\hline
		\end{tabular}
	
		\begin{tabular}{|l|l|l|l|}
			\hline
			Dezimal & Bin�r & Oktal & Hexadeximal\\
			\hline
			\hline
			5 & 0101 & 5 & 5\\
			\hline
			6 & 0110 & 6 & 6\\
			\hline
			7 & 0111 & 7 & 7\\
			\hline
			8 & 1000 & 10 & 8\\
			\hline
			9 & 1001 & 11 & 9\\
			\hline
		\end{tabular}
	
		\begin{tabular}{|l|l|l|l|}
			\hline
			Dezimal & Bin�r & Oktal & Hexadeximal\\
			\hline
			\hline
			10 & 1010 & 12 & A\\
			\hline
			11 & 1011 & 13 & B\\
			\hline
			12 & 1100 & 14 & C\\
			\hline
			13 & 1101 & 15 & D\\
			\hline
			14 & 1110 & 16 & E\\
			\hline
			15 & 1111 & 17 & F\\
			\hline
		\end{tabular}
	\end{multicols}