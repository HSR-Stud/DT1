\section{Schaltalgebra}
\subsection{Rechenregeln}

\textbf{Allg. $\Rightarrow$ Punkt vor Strich oder $\land$ vor $\lor$}

	\begin{tabular}{llll}
		Verkn"upfung mit 0 & $ a \lor 0 = a $ & $ a \land 0 = 0 $ & $ a \not= 0 = a $\\
		Verkn"upfung mit 1 & $ a \lor 1 = 1 $ & $ a \land 1 = a $ & $ a \not= 1 = \overline{a} $ \\
		Verkn. mit sich selbst & $ a \lor a = a $ & $ a \land a = a $ & $ a \not= a = 0 $ \\
		Verkn. mit Inversem & $ a \lor \overline{a} = 1 $ & $ a \land \overline{a} = 0 $ & $ a \not= \overline{a} = 1 $ \\
		\\
		Kommutativgesetz & $ a \lor b = b \lor a $ & $ a \land b = b \land a $ & $ a \not= b = b \not= a $\\
		Assioziativgesetz & $ (a \lor b) \lor c = a \lor (b \lor c) $ & $ (a \land b) \land c = a \land (b \land c) $ & $ (a \not= b) \not= c = a \not= (b \not= c) $ \\
		Distributivgesetz & $ a \land (b \lor c) = (a \land b) \lor (a \land c) $ & $ a \lor (b \land c) = (a \lor b) \land (a \lor c) $ & $ a \land (b \not= c) = (a \land b) \not= (a \land c) $ \\	
		\end{tabular}
\subsection{Vereinfachungen}
\begin{tabular}{lllll}
	$ a \lor (a \land b) = a $ & $(a \land b) \lor (a \land \overline{b}) = a $ &\hspace{2.0cm} &
	$  (a \land\overline{b}) \lor b = a \lor b$ & $ (a \land \overline{b}) \not= b = a \lor b $\\
	$  a \land (a \lor b) = a $ & $ (a \lor b) \land (a \lor \overline{b}) = a $ &\hspace{2.0cm}&
	$ (a \lor \overline{b}) \land b = a \land b $ &$(a \not= \overline{b}) \land b = a \land b  $\\
\end{tabular}

\subsection{Shannon und DeMorgan}
\begin{multicols}{2}
	\textbf{DeMorgan}\\
	$a \land b = \overline{\overline{(a \land b)}}= \overline{(\overline{a} \lor \overline{b})}$\\
	$a \lor b = \overline{\overline{(a \lor b)}} = \overline{(\overline{a} \land \overline{b})}$

	\textbf{Shanon} (Vereinfachter Demorgan)\\
	- alle Variablen negieren\\
	- alle Operatoren negieren ($\lor \rightarrow \land / \land \rightarrow \lor$)\\
	- ganzer Ausdruck negieren
\end{multicols}
\begin{tabular}{lll}
	Ursprungsschaltung: & Shannon & DeMorgan\\
		\includegraphics[width=0.3\textwidth]{pics/shanonursprung} & 
		\includegraphics[width=0.3\textwidth]{pics/shanonende} &
		\includegraphics[width=0.3\textwidth]{pics/demorganende}\\
\end{tabular}


\subsection{Wahrheitstabelle, KDNF, KKNF}
\begin{multicols}{2}
\subsubsection{KDNF (Kanonisch  Disjunktive Normalform)}
\vspace{-5pt}
\textbf{Minterm}$~~$ \\
Zeile in WHT mit Funktionswert = \textbf{1}\\
$\rightarrow$ Eingangsvariablen AND($\land$)-Verknüpfen \\
$\rightarrow $ Eingangsvariable mit Wert \textbf{0} invertieren\\
\textbf{DNF (Sums of products)}\\
Minterme OR($\lor$)-Verknüpfen\\
\textbf{Kurzschreibweise:}\\
DNF: $ b = \lor([1],[3],5) $ \\
Zahl entspricht Zeile, Eckige Klammern bedeutet don't care.\\ 

\subsubsection{KKNF (Kanonisch Konjunktive Normalform)}
\vspace{-5pt}
\textbf{Maxterm}$~~$ \\
Zeile in WHT mit Funktionswert = \textbf{0}\\
$\rightarrow$ Eingangsvariablen OR($\lor$)-Verknüpfen \\
$\rightarrow $ Eingangsvariable mit Wert \textbf{1} invertieren\\
\textbf{KNF (Products of sums)}\\
Maxterme AND($\land$)-Verknüpfen\\
\textbf{Kurzschreibweise:}\\
KNF: $ a = \land([1],[2],3) $ \\
Zahl entspricht Zeile, Eckige Klammern bedeutet don't care.\\ 

\includegraphics[width=0.5\textwidth]{pics/KNFDNF}
\end{multicols}

\subsection{Karnaugh-Diagramm}
\begin{tabular}{lll}
	\includegraphics[width=0.1\textwidth]{pics/kv/2erKV} & 
	\includegraphics[width=0.1\textwidth]{pics/kv/3erKV} &
	\includegraphics[width=0.15\textwidth]{pics/kv/4erKV}\\
\end{tabular}
\subsection{Arbeiten mit KV-Diagramm}
\begin{enumerate}
\setlength{\itemsep}{1pt}
  \setlength{\parskip}{0pt}
  \setlength{\parsep}{0pt}
\item Aufstellen der Wahrheitstabelle\\
\item "Ubertragen der Funktionswerte der Wahrheitstabelle in KV Diagramm\\
\item M"oglichst grosse Gruppen à $2^n$ Felder bilden\\
\begin{tabular}{|p{7cm}|p{7cm}}
	\textbf{DNF} & \textbf{KNF} \\
	Gruppen von Feldern mit Wert 1 oder d & Gruppen von Feldern mit Wert 0 oder d\\
\end{tabular}
\item Terme aus KV-Diagramm lesen\\
\begin{tabular}{|p{7cm}|p{7cm}}
	Variablen in Gruppe \textbf{AND}-Verknnüpfen & Variablen in Gruppe \textbf{OR}-Verknüpfen\\
	$\rightarrow$ \textbf{Variable = 0 invertieren}&$\rightarrow$ \textbf{Variable = 1 invertieren}\\
	\textbf{OR}-Verknüpfen aller Primimplikanten & \textbf{AND}-Verknüpfen aller Primimplikanten\\
\end{tabular}
\end{enumerate}